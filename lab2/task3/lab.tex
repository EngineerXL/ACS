\section{Приведение уравнения поверхности второго рода к каноническому виду}

\begin{sagesilent}
    # Точность вычислений (знаков после запятой)
    acc = 10

    # Границы для трёхмерного графика
    plot_range = 10.0

    u(x, y, z) = 11 * x ^ 2 - 2 * x * y - 2 * x * z + 2 * y * z + 9 * z ^ 2 - 4 * x + y + z
\end{sagesilent}

Исходное уравнение поверхности второго рода:

$\sage{u(x, y, z)} = 0$

\begin{sagesilent}
    # Из уравнения поверхности получим коэффициенты
    a11 = 11
    a22 = 0
    a33 = 9

    a12 = -2 / 2
    a13 = -2 / 2
    a23 = 2 / 2

    a0 = 0
    a1 = -4 / 2
    a2 = 1 / 2
    a3 = 1 / 2

    matrA = matrix([
    [a11, a12, a13],
    [a12, a22, a23],
    [a13, a23, a33]
    ])

    vecA = vector([a1, a2, a3])
\end{sagesilent}

Матричный вид уравнения:

$\sage{LatexExpr("A = " + latex(matrA))}$

И вектор линейных коэффициентов:

$\sage{LatexExpr("a = " + latex(vecA))}$

\begin{sagesilent}
    # Характеристический многочлен
    var("z")
    matrAm = matrA - z * identity_matrix(3)
    detA = det(matrAm).simplify_full()
\end{sagesilent}

Составим характеристический многочлен:

$\sage{LatexExpr(r'\det(A - z * E) = ' + latex(detA))}$

\begin{sagesilent}
    # Собственные числа
    solutions = solve(detA == 0, z)
    eigenvalues = []
    for i in range(len(solutions)):
        if (solutions[i].rhs().imag().n(digits = acc) < 10 ** -acc):
            eigenvalues.append(solutions[i].rhs().real().n(digits = acc))
\end{sagesilent}

Корни характеристического уравнения:

$\sage{LatexExpr("z_0 = " + str(eigenvalues[0]) + r', z_1 = ' + str(eigenvalues[1]) + r', z_2 = ' + str(eigenvalues[2]))}$

\begin{sagesilent}
    # Собственные вектора
    eigenvectors = []
    var("x1 x2 x3")
    for i in range(len(eigenvalues)):
        matrAmzE = matrA - eigenvalues[i] * identity_matrix(3)
        system_of_equations = []
        for j in range(3):
            system_of_equations.append(matrAmzE[j][0] * x1 + matrAmzE[j][1] * x2 + matrAmzE[j][2] * x3 == 0)
        # Решаем систему уравнений методом Гаусса
        system_of_equations[1] = system_of_equations[1] - system_of_equations[0] * matrAmzE[1][0] / matrAmzE[0][0]
        system_of_equations[2] = system_of_equations[2] - system_of_equations[0] * matrAmzE[2][0] / matrAmzE[0][0]
        system_of_equations[2] = system_of_equations[2] - system_of_equations[1] * (system_of_equations[2].lhs() / system_of_equations[1].lhs())
        # Проверяем, что ранг матрицы меньше трёх
        if (system_of_equations[2].lhs() == 0 and system_of_equations[2].rhs() == 0):
            system_of_equations[2] = (x3 == 1)
        else:
            print("Ранг матрицы равен трём")
        eigenvec = vector([j.rhs().n(digits = acc) for j in solve(system_of_equations, x1, x2, x3)[0]])
        ans = matrAmzE * eigenvec
        eigenvectors.append(eigenvec.n(digits = acc))
\end{sagesilent}

Получим собственные векторы:

$\sage{LatexExpr(r'\lambda_0 = ' + str(eigenvalues[0].n(digits = acc)) + ', s_0 = ' + str(eigenvectors[0].n(digits = acc)))}$

$\sage{LatexExpr(r'\lambda_1 = ' + str(eigenvalues[1].n(digits = acc)) + ', s_1 = ' + str(eigenvectors[1].n(digits = acc)))}$

$\sage{LatexExpr(r'\lambda_2 = ' + str(eigenvalues[2].n(digits = acc)) + ', s_2 = ' + str(eigenvectors[2].n(digits = acc)))}$

\begin{sagesilent}
    # Получим матрицу перехода
    matrS = matrix(eigenvectors)
\end{sagesilent}

Получим матрицу перехода к диагональной матрице

$\sage{LatexExpr("S = " + latex(matrS.n(digits = acc)))}$

\begin{sagesilent}
    # Нормируем собсветнные векторы
    for i in range(len(eigenvectors)):
        eigenvectors[i] = eigenvectors[i] / sqrt(eigenvectors[i].dot_product(eigenvectors[i]))
    matrS = matrix(eigenvectors)
\end{sagesilent}

После нормировки собственных векторов матрица перехода имеет вид:

$\sage{LatexExpr("S^* = " + latex(matrS.n(digits = acc)))}$

\begin{sagesilent}
    matrAdiag = matrS * matrA * matrS.T
\end{sagesilent}

Приводя к диагональному виду:

$\sage{LatexExpr("S^{*} * A * {S^{*}}^T = " + latex(matrAdiag.n(digits = acc)))}$

\begin{sagesilent}
    vecAnew = matrS.T * vecA
\end{sagesilent}

Новый вектор линейных коэффициентов:

$\sage{vecAnew}$

\begin{sagesilent}
    v(x, y, z) = 0
    variables = [x, y, z]
    for i in range(len(variables)):
        v += matrAdiag[i][i] * variables[i] ^ 2 + variables[i] * vecAnew[i]
\end{sagesilent}

После приведения к каноническому виду получим:

$\sage{v(x, y, z) == 0}$

\begin{sagesilent}
    # Сделаем замену переменных
    coef_linear = []
    for i in range(len(vecAnew)):
        coef_linear.append((vecAnew[i] / (2 * matrAdiag[i][i])) ** 2)
        if (matrAdiag[i][i] * vecAnew[i] < 0):
            coef_linear[-1] = -coef_linear[-1]
    v(x, y, z) = a0
    variables = [x, y, z]
    for i in range(len(variables)):
        v += matrAdiag[i][i] * (variables[i] + coef_linear[i]) ^ 2 - matrAdiag[i][i] * abs(coef_linear[i])
\end{sagesilent}

Сделаем замену переменных, получим:

$\sage{v(x, y, z) == 0}$

\pagebreak

Сравним графики исходного уравнения поверхности второго рода и канонического уравнения после замены переменных.

График функции $\sage{u(x, y, z)} = 0$
\begin{center}
  \sageplot[trim=50 50 50 50, clip, width=10cm][png]{implicit_plot3d(u, (-plot_range, plot_range), (-plot_range, plot_range), (-plot_range, plot_range), color="orange")}
\end{center}

График функции $\sage{v(x, y, z)} = 0$
\begin{center}
  \sageplot[trim=50 50 50 50, clip, width=10cm][png]{implicit_plot3d(v, (-plot_range, plot_range), (-plot_range, plot_range), (-plot_range, plot_range), color="orange")}
\end{center}
